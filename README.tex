% Options for packages loaded elsewhere
\PassOptionsToPackage{unicode}{hyperref}
\PassOptionsToPackage{hyphens}{url}
%
\documentclass[
]{article}
\usepackage{lmodern}
\usepackage{amssymb,amsmath}
\usepackage{ifxetex,ifluatex}
\ifnum 0\ifxetex 1\fi\ifluatex 1\fi=0 % if pdftex
  \usepackage[T1]{fontenc}
  \usepackage[utf8]{inputenc}
  \usepackage{textcomp} % provide euro and other symbols
\else % if luatex or xetex
  \usepackage{unicode-math}
  \defaultfontfeatures{Scale=MatchLowercase}
  \defaultfontfeatures[\rmfamily]{Ligatures=TeX,Scale=1}
\fi
% Use upquote if available, for straight quotes in verbatim environments
\IfFileExists{upquote.sty}{\usepackage{upquote}}{}
\IfFileExists{microtype.sty}{% use microtype if available
  \usepackage[]{microtype}
  \UseMicrotypeSet[protrusion]{basicmath} % disable protrusion for tt fonts
}{}
\makeatletter
\@ifundefined{KOMAClassName}{% if non-KOMA class
  \IfFileExists{parskip.sty}{%
    \usepackage{parskip}
  }{% else
    \setlength{\parindent}{0pt}
    \setlength{\parskip}{6pt plus 2pt minus 1pt}}
}{% if KOMA class
  \KOMAoptions{parskip=half}}
\makeatother
\usepackage{xcolor}
\IfFileExists{xurl.sty}{\usepackage{xurl}}{} % add URL line breaks if available
\IfFileExists{bookmark.sty}{\usepackage{bookmark}}{\usepackage{hyperref}}
\hypersetup{
  hidelinks,
  pdfcreator={LaTeX via pandoc}}
\urlstyle{same} % disable monospaced font for URLs
\setlength{\emergencystretch}{3em} % prevent overfull lines
\providecommand{\tightlist}{%
  \setlength{\itemsep}{0pt}\setlength{\parskip}{0pt}}
\setcounter{secnumdepth}{-\maxdimen} % remove section numbering

\date{}

\begin{document}

\hypertarget{first-order-logic-parser---documentation}{%
\section{First Order Logic Parser -
Documentation}\label{first-order-logic-parser---documentation}}

\hypertarget{dependencies-and-installation}{%
\subsection{Dependencies and
Installation}\label{dependencies-and-installation}}

First and foremost, the parser requires Python version 3.6+ to run. It
has been developed and tested on 3.7.x and 3.8.x

The parser relies on three imported packages from the Python standard
library, \texttt{re}, \texttt{pathlib} and \texttt{sys}, and one
external python package, \texttt{graphviz}.

The former three do not need special installation.

To use \texttt{graphviz}, please follow the instructions at the
beginning of the following webpage:
https://graphviz.readthedocs.io/en/stable/manual.html

In case of not being able to access this page, the instructions are
paraphrased below: - Please first install Graphviz on your operating
system, by choosing a download from the following webpage:
https://www.graphviz.org/download/

\begin{itemize}
\item
  Once Graphviz is installed, please make sure your
  \texttt{/graphviz/bin/} directory is added to your system's
  \texttt{PATH} variable, so that the \texttt{graphviz} Python package
  can successfully use Graphviz's functions from the command line under
  the hood. You can check if this was successful by running
  \texttt{dot\ -V} from your command line outside of the folder in
  question - if done correctly, this command will print the version of
  your Graphviz installation.
\item
  To finalise installation, run \texttt{pip\ install\ graphviz} from
  your command line.
\end{itemize}

\hypertarget{usage}{%
\subsection{Usage}\label{usage}}

First make sure your input file \texttt{filename.ext} is in the same
working directory as the program \texttt{parser.py}.

To use the program, run the command
\texttt{python\ ./parser.py\ ./filename.ext} from the directory
containing the parser and the input file.

If the program \texttt{parser.py} is not called via the command line
with the filename as an argument following it, it will not know what
input to parse, and will exit.

\hypertarget{input}{%
\subsubsection{Input}\label{input}}

The parser expects as input a text file of at least 7 lines, with
exactly 7 of these lines beginning with \texttt{setname:} to define some
features of the grammar. Guidance supplied within the coursework
specification and in the DUO FAQ on how to format the sets, which sets
are allowed to be empty and what characters are allowed in members of
each set has been followed - for example, the input will be invalid if
members of a set are separated by anything but a single space, or if
there are fewer than 5 entries on the ``connectives:'' line, of if the
formula is split over multiple lines that are not consecutive.

\hypertarget{general-program-flow}{%
\subsubsection{General Program Flow}\label{general-program-flow}}

The parser takes an input file and ingests each line, making sense of
the members of each set in the grammar and storing them accordingly. It
then prints the grammar to a file (or, if for some reason it cannot
print to a file, displays the grammar to the console) before starting to
parse the token stream. It uses a simple recursive predictive parsing
technique as described in lectures, taking advantage of the LL(k) nature
of the First Order production rules. While parsing, the program builds
the parse tree symbol by symbol. It finishes by rendering the tree and
outputting a success message to the log file.

\hypertarget{outputs}{%
\subsubsection{Outputs}\label{outputs}}

If ran on a valid file, the parser will produce the following outputs:

\begin{itemize}
\tightlist
\item
  \texttt{grammar.txt}, a file describing the grammar of the language
  supplied in the input file
\item
  \texttt{log.log}, a text file containing a message informing the user
  of successful parsing of the input file
\item
  \texttt{ParseTree}, a side effect file of Graphviz's graph rendering
  that can be ignored
\item
  \texttt{ParseTree.pdf}, a pdf containing the visual representation of
  the parse tree for the formula
\end{itemize}

In case of an error, \texttt{log.log} will always contain the error
message that provides more detail on why the parser halted.

Depending on where the error occurred, the program may not have output
the \texttt{ParseTree} and \texttt{ParseTree.pdf} files, say if the
error is in the formula but the rest of the file was formatted
correctly, or it may have also not output \texttt{grammar.txt}, say if
there was a problem with the file formatting or the entries in a set of
the grammar.

The parse tree is of a standard construction and aesthetic. Internal
nodes, denoted by single bold letters, are non-terminal symbols, while
leaves are terminal symbols.

\hypertarget{examples}{%
\subsection{Examples}\label{examples}}

Contained within the \texttt{examples} folder in the submission are a
few examples of the program's outputs based on some devised inputs.

\begin{itemize}
\tightlist
\item
  \texttt{ex1} contains the outputs from a typical valid file, similar
  to the one provided with the coursework
\item
  \texttt{ex2} contains the outputs from a file that has a valid
  grammar, but an error in its formula
\item
  \texttt{ex3} contains the outputs from a file that has a valid
  formula, but an error in one of the predicate definitions
\end{itemize}

\end{document}
